\documentclass[oneside]{book}
\usepackage[utf8]{inputenc}
\usepackage{xeCJK}
\usepackage{bm}
\usepackage{geometry}
\usepackage{fontspec} 
\usepackage[normalem]{ulem}
\setCJKmainfont{PingFangSC-Regular}
\setCJKmonofont{PingFangSC-Regular}
\font\titlteFont=NovaMono at 40pt
\font\authorFont=NovaMono at 24pt
\title{\titlteFont Machine Learning Notes}
\author{\authorFont Jiang Tao}
\date{}
\begin{document}
	\begin{titlepage}
		\maketitle
	\end{titlepage}
	\tableofcontents
	\chapter*{机器学习的主要内容及其所需知识}
	 	\section{主要内容}
			\begin{enumerate} 
				\item 数据处理: 对数据进行降维(Dimensionality Reduction) 和特征缩放(Feature Scaling) 等.
				\item 模型定义: 线性模型, 神经网络, SVM 等.
				\item 定义损失函数: 根据不同的模型, 找出适合它的损失函数.
				\item 最优化: 梯度下降等.
				\item 模型评估: 对模型进行评估与预防过拟合
			\end{enumerate}
		\section{所需知识}
			\subsection{线性代数}
			\subsection{概率论}
			\subsection{信息论}
	\chapter{数据处理}
		\section{特征缩放(Feature Scaling)}
			特征缩放的作用: 
		\section{数据降维(Dimensionality Reduction)}
		
	\chapter{模型定义}
	\chapter{最优化}
	\chapter{模型评估}
\end{document}